\documentclass[12pt,a4paper,oneside]{article}

\usepackage[QX]{polski}

\usepackage[utf8]{inputenc}
\usepackage{latexsym}
\usepackage{tgpagella}
\usepackage{lmodern}
\usepackage{amsmath,amsthm,amsfonts,amssymb,alltt}
\usepackage{epsfig}
\usepackage{pdflscape}
\usepackage{caption}
\usepackage{indentfirst}
\usepackage{float}
%\usepackage{showkeys}
\bibliographystyle{plabbrv}


\usepackage{color}
\usepackage[polish]{babel}
\usepackage{datetime2}
\usepackage[x11names,dvipsnames,table]{xcolor}
\usepackage{hyperref}
\hypersetup{
pdfauthor={Roman Czapla, Olaf Bar},
colorlinks=True,
linkcolor=darkgray,  % color of internal links (change box color with linkbordercolor)
citecolor=BrickRed,  % color of links to bibliography
filecolor=Magenta,   % color of file links
urlcolor=BlueViolet}	%%pdfpagemode=FullScreen}

% diagramy, grafy itp.
\usepackage{tikz}
\usetikzlibrary{positioning}
\usetikzlibrary{arrows}
\usetikzlibrary{arrows.meta}
\usetikzlibrary{chains,fit,shapes,calc}
\tikzset{main node/.style={circle,fill=blue!20,draw,minimum size=1cm,inner sep=0pt}}

% algorytmy
\usepackage[linesnumbered,lined,commentsnumbered]{algorithm2e}
\SetKwFor{ForEach}{for each}{do}{end for}%
\SetKwFor{ForAll}{for all}{do}{end for}%
\newenvironment{myalgorithm}
{\rule{\textwidth}{0.5mm}\\\SetAlCapSty{}\SetAlgoNoEnd\SetAlgoNoLine\begin{algorithm}}{\end{algorithm}\rule{\textwidth}{0.5mm}}


%---------------------
\overfullrule=2mm
\pagestyle{plain}
\textwidth=15cm \textheight=685pt \topmargin=-25pt \linespread{1.3} 
\setlength{\parskip}{0pt}
\setlength\arraycolsep{2pt}
\oddsidemargin = 0.9cm
\evensidemargin =-0.1cm

\captionsetup{width=.95\linewidth, justification=centering}
%---------------------




\newtheorem{tw}{Twierdzenie}[section]
\newtheorem{lem}[tw]{Lemat}
\newtheorem{co}[tw]{Wniosek}
\newtheorem{prop}[tw]{Stwierdzenie}
\theoremstyle{definition}
\newtheorem{ex}{Przykład}
\newtheorem{re}[tw]{Uwaga}
\newtheorem{de}{Definicja}[section]



\newcommand{\bC}{{\mathbb C}}
\newcommand{\bR}{{\mathbb R}}
\newcommand{\bZ}{{\mathbb Z}}
\newcommand{\bQ}{{\mathbb Q}}
\newcommand{\bN}{{\mathbb N}}
\newcommand{\captionT}[1]{\caption{\textsc{\footnotesize{#1}}}}
\renewcommand\figurename{Rys.}

\numberwithin{equation}{section}
\renewcommand{\thefootnote}{\arabic{footnote})}
%\renewcommand{\thefootnote}{\alph{footnote})}



\begin{document}
% --------------------------------------------
% Strona tytułowa
% --------------------------------------------

\thispagestyle{empty}
\begin{titlepage}
\begin{center}\Large
Uniwersytet Komisji Edukacji Narodowej w Krakowie\\
\large
Instytut Bezpieczeństwa i Informatyki\\
\vskip 10pt
\end{center}
\begin{center}
\centering \includegraphics[width=1.0\columnwidth]{images/logo.png}
\end{center}

\begin{center}
 {\bf \fontsize{14pt}{14pt}\selectfont PROJEKT INŻYNIERSKI\\ DOKUMENTACJA INSTALACYJNO-KONFIGURACYJNA SYSTEMU}
\end{center}
\vskip 5pt
\begin{center}
 {\bf \fontsize{22pt}{22pt}\selectfont RADAR ODCINKOWY}
\end{center}

\begin{center}
 {\fontsize{12pt}{12pt}\selectfont wykonany przez: }
\end{center}
\begin{center}
 {\bf\fontsize{16pt}{16pt}\selectfont Tomasz Górski}\\
 {\fontsize{12pt}{12pt}\selectfont Nr albumu: 151896 \\\&\\}
 {\bf\fontsize{16pt}{16pt}\selectfont Tomasz Joniec}\\
 {\fontsize{12pt}{12pt}\selectfont Nr albumu: 151861\\\&\\}
 {\bf\fontsize{16pt}{16pt}\selectfont Patryk Golonka}\\
 {\fontsize{12pt}{12pt}\selectfont Nr albumu: 145857}
\end{center}
\begin{center}
 {\fontsize{12pt}{12pt}\selectfont pod opieką:}\\
 {\bf\fontsize{12pt}{12pt}\selectfont dr inż. Grzegorz Sokal, mgr Łukasz Przybytek}
\end{center}

%\mbox{}
\vspace*{\fill}
%\vskip 50pt
\begin{center}
\large
Kraków \the\year\\
(ostatnia aktualizacja: \DTMcurrenttime,\;\today)
\end{center}
\end{titlepage}
\setcounter{page}{0} 
\newpage\null\thispagestyle{empty}
%\setcounter{page}{0} 
%\newpage
%\thispagestyle{empty}

\tableofcontents


\newpage

\section{Szczegółowa dokumentacja przebiegu instalacji oraz konfiguracji systemu.}
\subsection{Wymagania sprzętowe}
Minimalne wymagania sprzętowe do poprawnego działania projektu:
\begin{itemize}
  \item  kamery IP ze wspraciem FTP oraz detekcją ruchu  
  \item  2 rdzenniowy procesor 
  \item  4 GB pamięci RAM 
  \item  dysk 32 GB
\end{itemize}
\subsection{Środowisko uruchomieniowe}
Do poprawnego działania wymagane jest instalacja poniższego oprogramowania:
\begin{itemize}
  \item  Ubuntu 20.04.6 LTS lub nowszy
  \item  Python 3.7
  \item  vsftpd 3.0.5
  \item  opencv-python 4.4.0.44
  \item  numpy 1.18.3
  \item  Apache 2.4
  \item  MySQL 8.0 (LTS)
  \item  PHP 8.2
  \item  TCPDF
\end{itemize}
\subsection{Opis instalacji}
\begin{enumerate}
  \item Przed konfiguracją klienta FTP na kamerze należy upewnić się, że usługa serwera FTP jest uruchomiona na terminalu z Ubuntu. 
  \item Kamera IP powinna zostać skonfigurowana w taki sposób aby detekcja samochodu powodowała zapisanie klatki i wysłanie jej za pomoca FTP do zdalej lokalizacji na systemie Ubuntu. W przypadku tego projektu kamera Zintronic umożliwa skonfigurowanie takiej opcji w zakładce 'Smart Event'.
  \item Do wybranej lokalizacji na dysku powinien zostać pobrany projekt za pomocą gh repo clone lub git clone.
  \item Zawartość folderu "apache" powinna znaleźć się w /var/www/html/.
  \item W plikach php należy zweryfikować zmienne 'host', 'username', 'password' (wymaganne do poprawnego połączenia z bazą danych). 
  \item W lokalizacji /var/www/html/ powinien być zainstalowany TCPDF.
  \item Po przeładowaniu serwera www za pomocą systemctl restart apache2 powinna pojawić się strona logowania pod przypisanym do niego adresem lokalnym. Domyślny login i hasło to admin:admin. 
\end{enumerate}




% --------------------------------------------------------------------
%%%%%%% odkomentować gdy bibliografia ma być wewnątrz dokumentu
% --------------------------------------------------------------------
%\begin{thebibliography}{11}
%
%\addcontentsline{toc}{section}{Literatura}
%
%\bibitem{ZAN}
%C. Zannoni and P. Pasini, 
%\emph{Advances in the Computer Simulatons of Liquid Crystals}, Kluwer Academic Publishers, 2000.
%
%\end{thebibliography}

\end{document}

